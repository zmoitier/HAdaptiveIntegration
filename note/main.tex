% !TeX root = main.tex
% LTeX: enabled=false

\documentclass[english,12pt]{article}


%*** \Usepackages =============================================
\usepackage[utf8]{inputenc}
\usepackage[T1]{fontenc}
\usepackage[a4paper,margin=2.5cm]{geometry}
\usepackage{amssymb,amsthm,mathtools}
\usepackage{graphicx}
\usepackage{babel}
\usepackage[colorlinks]{hyperref}
\usepackage{microtype}
% \usepackage{caption}
% \usepackage{subcaption}
% \usepackage{enumitem}
% \usepackage{tabularx}

%*** Links and drawing ----------------------------------------
% \usepackage{xcolor}
\usepackage{tikz}

%*** Math -----------------------------------------------------
\usepackage{mathrsfs,bm,siunitx}
\usepackage[capitalize,nameinlink]{cleveref}
\numberwithin{equation}{section}
\raggedbottom{}
\allowdisplaybreaks{}

%*** Bibliography ---------------------------------------------
\usepackage[backend=biber,style=alphabetic,sorting=nyt]{biblatex}
\addbibresource{references.bib}
\usepackage{csquotes}


%*** Colors ===================================================
%*** \definecolor ---------------------------------------------
\definecolor{Blue}{HTML}{1F77B4}
\definecolor{Orange}{HTML}{FF7F0E}
\definecolor{Green}{HTML}{2CA02C}
\definecolor{Red}{HTML}{D62728}
\definecolor{Grey}{HTML}{7F7F7F}
% \definecolor{Purple}{HTML}{9467BD}
% \definecolor{Brown}{HTML}{8C564B}
% \definecolor{Pink}{HTML}{E377C2}

\NewDocumentCommand{\Black}{m}{\textcolor{black}{#1}}
\NewDocumentCommand{\Blue}{m}{\textcolor{Blue}{#1}}
\NewDocumentCommand{\Orange}{m}{\textcolor{Orange}{#1}}
\NewDocumentCommand{\Green}{m}{\textcolor{Green}{#1}}
\NewDocumentCommand{\Red}{m}{\textcolor{Red}{#1}}
\NewDocumentCommand{\Grey}{m}{\textcolor{Grey}{#1}}
% \NewDocumentCommand{\Purple}{m}{\textcolor{Purple}{#1}}
% \NewDocumentCommand{\Brown}{m}{\textcolor{Brown}{#1}}
% \NewDocumentCommand{\Pink}{m}{\textcolor{Pink}{#1}}

\NewDocumentCommand{\todo}{m}{\noindent{\color{Red}\texttt{:\!:todo:\!:} \bfseries #1}}
\NewDocumentCommand{\zmcom}{m}{\noindent{\color{Green}\texttt{:\!:Z:\!:} #1}}

%*** hypersetup ----------------------------------------------
\hypersetup{%
    colorlinks = True,
    linkcolor  = Red,
    citecolor  = Green,
    urlcolor   = Blue,
}


%*** Math =====================================================
%*** mathbb ---------------------------------------------------
\NewDocumentCommand{\bbC}{}{\mathbb{C}}
\NewDocumentCommand{\bbN}{}{\mathbb{N}}
\NewDocumentCommand{\bbR}{}{\mathbb{R}}
\NewDocumentCommand{\bbZ}{}{\mathbb{Z}}

%*** mathcal --------------------------------------------------
\NewDocumentCommand{\calX}{}{\mathcal{X}}

%*** Constants ------------------------------------------------
\NewDocumentCommand{\ex}{}{\mathsf{e}}
\NewDocumentCommand{\im}{}{\mathsf{i}\mkern1mu}

%*** Spaces ---------------------------------------------------
\NewDocumentCommand{\spC}{}{\mathscr{C}}
\NewDocumentCommand{\spL}{}{\mathrm{L}}
\NewDocumentCommand{\spH}{}{\mathrm{H}}

\NewDocumentCommand{\spPolyP}{o}{\IfNoValueTF{#1}{\mathbb{P}}{\mathbb{P}_{#1}}}
\NewDocumentCommand{\spPolyH}{m}{\mathbb{H}_{#1}}
\NewDocumentCommand{\spPolyQ}{m}{\mathbb{Q}_{#1}}

%*** Keywords -------------------------------------------------
\NewDocumentCommand{\loc}{}{\mathrm{loc}}
\NewDocumentCommand{\comp}{}{\mathrm{comp}}

%*** Operators ------------------------------------------------
\NewDocumentCommand{\di}{m}{\mathop{}\!\mathrm{d}#1}
\DeclareMathOperator{\OO}{\mathcal{O}}
\DeclareMathOperator{\oo}{\mathcal{\scriptstyle{}O}}

%*** Delimiter ------------------------------------------------
\DeclarePairedDelimiter{\plr}\lparen\rparen%
\DeclarePairedDelimiter{\clr}\lbrack\rbrack%
\DeclarePairedDelimiter{\blr}\lbrace\rbrace%
\DeclarePairedDelimiter{\alr}\langle\rangle%
\DeclarePairedDelimiter{\abs}\lvert\rvert%
\DeclarePairedDelimiter{\norm}\lVert\rVert%

\DeclarePairedDelimiterXPP\normF[1]{}\lVert\rVert{_\mathrm{F}}{#1}

\DeclarePairedDelimiterX\setst[2]\lbrace\rbrace{#1\:\delimsize\vert\:\mathopen{}#2} % chktex 21
\DeclarePairedDelimiterX\setwt[2]\lbrace\rbrace{#1\:{:}\:\mathopen{}#2}

\DeclarePairedDelimiterX\ioo[2]\lparen\rparen{#1,\:#2}
\DeclarePairedDelimiterX\ioc[2]\lparen\rbrack{#1,\:#2}
\DeclarePairedDelimiterX\ico[2]\lbrack\rparen{#1,\:#2}
\DeclarePairedDelimiterX\icc[2]\lbrack\rbrack{#1,\:#2}

\NewDocumentCommand\restr{s m m}{%
    \IfBooleanTF{#1}{%
        \left. #2 \right\vert_{#3}%
    }{%
        #2 \arrowvert_{#3}%
    }%
}

%*** Specials -------------------------------------------------
\RenewDocumentCommand{\vec}{m}{\bm{#1}}
\NewDocumentCommand{\eps}{}{\varepsilon}

\NewDocumentCommand{\Simplex}{}{\triangle}
\NewDocumentCommand{\Orthotope}{}{\square}
\NewDocumentCommand{\St}{}{\mathsf{st}}
\NewDocumentCommand{\Rf}{}{\mathsf{ref}}

\NewDocumentCommand{\Quad}{}{\mathcal{Q}}
\NewDocumentCommand{\Err}{}{\mathcal{E}}
\NewDocumentCommand{\lo}{}{\mathsf{L}}
\NewDocumentCommand{\hi}{}{\mathsf{H}}


%*** Environment ==============================================
\theoremstyle{definition}
\newtheorem{definition}{Definition}[section]
\newtheorem{notation}[definition]{Notation}
\newtheorem{assumption}[definition]{Assumption}

\crefname{assumption}{Assumption}{Assumptions}
\Crefname{assumption}{Assumption}{Assumptions}

\theoremstyle{plain}
\newtheorem{lemma}[definition]{Lemma}
\newtheorem{theorem}[definition]{Theorem}
\newtheorem{corollary}[definition]{Corollary}

\theoremstyle{remark}
\newtheorem{remark}[definition]{Remark}
\newtheorem{example}[definition]{Example}


%*** Title ====================================================
\title{Note on adaptive integration on simplices and orthotopes}

\author{
    Luiz \textsc{Faria}\({}^1\)
    \and
    Zo{\"\i}s \textsc{Moitier}\({}^2\)
}

\date{\raggedright\footnotesize%
    \({^1}\)POEMS, CNRS, Inria, ENSTA Paris, Institut Polytechnique de Paris, 91120 Palaiseau, France.\\
    \({}^2\)IDEFIX, Inria, ENSTA Paris, Institut Polytechnique de Paris, 91120 Palaiseau, France.\\[1em]
    \large\Red{\textbf{\today}}
}

%*** Document =================================================
\begin{document}

\maketitle

\begin{abstract}
    Note on adaptive integration using embedded cubature rule on simplices and orthotopes.
\end{abstract}

\setcounter{tocdepth}{2}
\tableofcontents

%*** include ==================================================
% !TeX root = main.tex
% LTeX: language=en-US

%***===========================================================
\section{Integration domain}

%***-----------------------------------------------------------
\subsection{Simplex}

For \( n  \in \bbN \), an \( n \)-simplex is an \( n \)-dimensional polytope which is the convex hull of its \( n+1 \) vertices \( v_0, \ldots, v_n \in \bbR^n \).
In order to have a non-degenerate simplex, we assume that the \( n+1 \) vertices are independent, which means that the \( n \) vectors \( \blr{v_1 - v_0, \ldots, v_n - v_0} \) are linearly independent.

\begin{definition}
    We define the \emph{reference \( n \)-simplex}
    \[
        \Simplex_\Rf^n \coloneqq \setst*{u \in \bbR_+^n}{u_1 + \cdots + u_n \leq 1} \subset \bbR^n,
    \]
    which is also defined by being the convex-hull of the vector \( \plr{0, \ldots, 0} \) and of the \( n \) canonical basis vectors \( \plr{1, 0, \ldots, 0},\, \ldots,\, \plr{0, \ldots, 0, 1} \) in \( \bbR^n \).
    We drop the superscript \( n \) when it is clear from the context.
\end{definition}

Let \( \Simplex \) be an \( n \)-simplex defined by the points \( v_0, \ldots, v_n \) and \( \Phi_\Simplex \colon \Simplex_\Rf \to \Simplex \) the affine map from the reference simplex to the simplex \( \Simplex \) defined by \( \Phi_\Simplex \colon u \mapsto A_\Simplex u + v_0 \) where
\[
    A_\Simplex = \begin{pmatrix}
        v_1 - v_0 & \cdots & v_n - v_0
    \end{pmatrix}.
\]

%***-----------------------------------------------------------
\subsection{Orthotope}

For \( n  \in \bbN \), an \( n \)-orthotope is an \( n \)-dimensional polytope which is Cartesian product of intervals \( \icc{a_1}{b_1} \times \cdots \times \icc{a_n}{b_n} \).
In order to have a non-degenerate orthotope, we assume that \( a_i < b_i \), for all \( 1 \leq i \leq n \).

\begin{definition}
    We define the \emph{reference \( n \)-orthotope}
    \[
        \Orthotope_\Rf^n \coloneqq \icc{0}{1}^n \subset \bbR^n.
    \]
    We drop the superscript \( n \) when it is clear from the context.
\end{definition}

% !TeX root = main.tex
% LTeX: language=en-US

%***===========================================================
\section{Embedded cubature rule}

For \( \Omega \) an open domain in \( \mathbb{R}^n \), mainly the interior of the reference simplex \( \Simplex_\Rf \) or reference orthotope \( \Orthotope_\Rf \).
An embedded cubature \( \Quad = \plr{\Quad_\lo, \Quad_\hi} \) is composed of two nested cubature rules, one of low order \( \Quad_\lo = \plr{\calX_\lo, \vec{w}_\lo} \) and one of high order \( \Quad_\hi = \plr{\calX_\hi, \vec{w}_\hi} \) where
\[
    \calX_\lo \coloneqq \blr*{x_1, \ldots, x_{N_\lo}}
    \subset \calX_\hi \coloneqq \blr*{x_1, \ldots, x_{N_\lo}, \ldots, x_{N_\hi}}
    \subset K
\]
are sets of points with \( 1 \leq N_\lo < N_\hi \) and
\[
    \vec{w}_\lo \coloneqq \plr*{w_1^\lo, \ldots, w_{N_\lo}^\lo}
    \quad \text{and} \quad
    \vec{w}_\hi \coloneqq \plr*{w_1^\hi, \ldots, w_{N_\hi}^\hi}
\]
are the corresponding weights.
In addition, we assume that the order of the low order cubature rule is \( d_\lo \) and the order of the high order cubature rule is \( d_\hi \) with \( d_\lo < d_\hi \).
For a continuous function \( f \in \spC\plr{\Omega} \), an embedded cubature rule is used to give an approximation \( \Quad\clr{f} \) of the integral of a function \( f \) over \( \Omega \) and give error estimate \( \Err\clr{f} \) of the approximation by
\[
    \Quad\clr{f} \coloneqq \Quad_\hi\clr{f} \approx \int_\Omega f\plr{x} \di{x}
    \qquad \text{and} \qquad
    \Err\clr{f} \coloneqq \abs*{\Quad_\hi\clr{f} - \Quad_\lo\clr{f}},
\]
where
\[
    \Quad_\lo\clr{f} \coloneqq \sum_{i=1}^{N_\lo} w_i^\lo f\plr*{x_i}
    \quad \text{and} \quad
    \Quad_\hi\clr{f} \coloneqq \sum_{i=1}^{N_\hi} w_i^\hi f\plr*{x_i}.
\]

\begin{align*}
    \mathbb{E}_\lo & \coloneqq \setst*{p \in \spPolyP}{\Quad_\lo\clr{p} = \int_\Omega p\plr{x} \di{x}} \supset \spPolyP[d_\lo],
    \\
    \mathbb{E}_\hi & \coloneqq \setst*{p \in \spPolyP}{\Quad_\hi\clr{p} = \int_\Omega p\plr{x} \di{x}} \supset \spPolyP[d_\hi].
\end{align*}

% !TeX root = main.tex
% LTeX: language=en-US

%***===========================================================
\section{Example of embedded quadrature rule}

%***-----------------------------------------------------------
\subsection{Simplex}

Moments:
\[
    \int_{\Simplex_\Rf^n} \vec{x}^{\vec{\alpha}} \di{\vec{x}} = \frac{\vec{\alpha}!}{\plr*{\abs{\vec{\alpha}}_1 + n}!}.
\]

One point cubature of order 1:
\[
    \vec{x}_* = \frac{1}{n+1} \vec{1}
    \quad \text{and} \quad
    w_* = \frac{1}{n!}
\]

\( x_*, x_0, x_1, \ldots, x_n \) where
\begin{align*}
    x_0 & = \plr*{0, \ldots, 0}
    \\
    x_1 & = \plr*{1, 0, \ldots, 0}
    \\
        & \vdots
    \\
    x_n & = \plr*{0, \ldots, 0, 1}
\end{align*}

\begin{align*}
    w_* + (n+1) w_1               & = \frac{1}{n!}
    \\
    \frac{w_*}{n+1} + w_1         & = \frac{1}{(n+1)!}
    \\
    \frac{w_*}{\plr{n+1}^2} + w_1 & = \frac{2}{(n+2)!}
    \\
    \frac{w_*}{\plr{n+1}^2}       & = \frac{1}{(n+2)!}
\end{align*}

\begin{align*}
    w_* & = \frac{\plr{n+1}^2}{\plr{n+2}!}
    \\
    w_1 & = \frac{1}{\plr{n+2}!}
\end{align*}

%***-----------------------------------------------------------
\subsection{Orthotope}

Moments:
\[
    \int_{\Orthotope_\Rf^n} \vec{x}^{\vec{\alpha}} \di{\vec{x}} = \prod_{i=1}^n \frac{1}{\alpha_i + 1}
\]

\[
    \vec{x}_0 = \frac{1}{2} \vec{1}
\]

One point cubature of order 1:
\[
    \vec{x}_* = \frac{1}{2} \vec{1}
    \quad \text{and} \quad
    w_* = 1
\]

\( x_*, x_0, x_1, \ldots, x_{2^n-1} \) where
\begin{align*}
    x_i = \plr{\text{binary representation}}
\end{align*}

\begin{align*}
    w_* + 2^n w_1               & = 1
    \\
    \frac{w_*}{2} + 2^{n-1} w_1 & = \frac{1}{2}
    \\
    \frac{w_*}{4} + 2^{n-1} w_1 & = \frac{1}{3}
    \\
    \frac{w_*}{4} + 2^{n-2} w_1 & = \frac{1}{4}
\end{align*}

\begin{align*}
    w_* + 2^n w_1     & = 1
    \\
    w_* + 2^{n+1} w_1 & = \frac{4}{3}
\end{align*}

\begin{align*}
    w_* & = \frac{2}{3}
    \\
    w_1 & = \frac{1}{2^n 3}
\end{align*}


%***===========================================================
% \section{Acknowledgement}


%***===========================================================
% \section{Funding}


%*** Bibliography =============================================
% \printbibliography%[heading=bibintoc,title={References}]


%*** Appendix =================================================
% \appendix

%***===========================================================
% \section{Miscellaneous}

\end{document}
